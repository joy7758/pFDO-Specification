\section{Adversarial Robustness: Defeating ``Semantic Vacuum''}
To address the concern regarding potential semantic degradation following the unlearning process, we conducted a \textbf{Logic Integrity Stress Test} on the RLCP protocol.

\subsection{Axiomatic Core vs. Volatile Facts}
We partitioned the A-FDO memory space into a protected \textbf{Axiomatic Core} (storing policy enforcement logic) and a \textbf{Volatile Fact} pool (initially 1,000,000 entries). A \textbf{99\% metabolic purge} was executed to simulate extreme unlearning.

\subsection{Experimental Results and Consistency}
The empirical data (see Table \ref{tab:robustness}) demonstrates that the logical accuracy remains at an absolute \textbf{100.0\%} despite the near-total loss of factual context. The execution latency exhibited a negligible deviation of only \textbf{0.02 $\mu$s}, decreasing from 2.0339 $\mu$s to 2.0127 $\mu$s. This confirms that the A-FDO logic core is physically and logically decoupled from factual redundancy, effectively negating the ``Semantic Vacuum'' hypothesis.

\begin{table}[h]
\centering
\caption{Adversarial Metabolic Test: Logic Consistency vs. Data Loss}
\begin{tabular}{|l|c|c|}
\hline
\textbf{Metric} & \textbf{Pre-Metabolism (100\% Facts)} & \textbf{Post-Metabolism (1\% Facts)} \\ \hline
Logic Accuracy  & 100.0\%                 & 100.0\%                  \\ \hline
Latency ($\mu$s)& 2.0339                  & 2.0127                   \\ \hline
\end{tabular}
\label{tab:robustness}
\end{table}
