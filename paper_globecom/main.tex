\documentclass[conference]{IEEEtran}
\usepackage{cite}
\usepackage{amsmath,amssymb,amsfonts}
\usepackage{algorithmic}
\usepackage{graphicx}
\usepackage{textcomp}
\usepackage{xcolor}
\usepackage{url}
\usepackage{booktabs}

\begin{document}

\title{PHY-Level Deterministic Control: Overlap Execution within IEEE 802.3dj RS-FEC Decode Window}

\author{\IEEEauthorblockN{Anonymous Author(s)}
\IEEEauthorblockA{\textit{Affiliation} \\
City, Country \\
email@domain.com}
}

\maketitle

\begin{abstract}
Next-generation Ethernet standards, such as IEEE 802.3dj (800 GbE/1.6 TbE), introduce significant latency variability due to complex Reed-Solomon Forward Error Correction (RS-FEC) schemes required for PAM4 signaling. Traditional network governance, operating at Layer 3 or above, fails to provide deterministic guarantees in these high-speed environments because policy enforcement occurs after frame reassembly, accumulating jitter. This paper proposes a PHY-level deterministic control mechanism anchored at the PMA-PCS sublayer boundary. By exploiting the RS-FEC decode window, we introduce an "overlap execution" strategy where policy lookup is performed in parallel with FEC decoding. We present the Active Governance Header (AGH) and Multi-stage Bit Vector (MsBV) lookup as implementation artifacts that enable $O(1)$ decision latency ($\approx 1.18 \mu s$), fitting strictly within the FEC latency budget. This cross-layer approach ensures that control decisions are finalized before the packet is fully reconstructed, eliminating serialization delay penalties and providing clock-cycle deterministic reliability for time-sensitive networks.
\end{abstract}

\begin{IEEEkeywords}
IEEE 802.3dj, RS-FEC, Deterministic Networking, PHY Control, Active Governance Header.
\end{IEEEkeywords}

\section{Introduction}
As Ethernet speeds scale to 800 Gbps and 1.6 Tbps under the developing IEEE 802.3dj standard, the physical layer (PHY) complexity increases dramatically. To combat signal degradation in PAM4 modulation, heavy Reed-Solomon Forward Error Correction (RS-FEC) codes are mandatory. While essential for signal integrity, these FEC schemes introduce significant, often variable, latency due to block buffering and iterative decoding.

The fundamental problem is that traditional network governance and flow control mechanisms operate at the MAC layer or higher (L3/L4). At terabit speeds, the time spent waiting for a frame to pass through the FEC decoder, PCS descrambler, and MAC reassembly creates a "governance lag." Decisions made after full frame reception are too late to prevent buffer congestion or enforce strict timing guarantees required by industrial and mission-critical applications.

To address this, we propose anchoring the governance control plane deep within the PHY, specifically at the Physical Medium Attachment (PMA) to Physical Coding Sublayer (PCS) boundary. We introduce a mechanism for **overlap execution**, where policy enforcement logic runs in parallel with the mandatory RS-FEC decoding latency. By utilizing a fixed-length Active Governance Header (AGH) detectable at the symbol level, we ensure that the "drop/forward" decision is ready the moment the FEC block is corrected, effectively hiding the governance latency within the unavoidable physical layer overhead.

\section{System Model: PHY-Level Governance Anchor}
We consider a network architecture compliant with IEEE 802.3dj, focusing on the receiver datapath.

\subsection{Latency Constraints and The FEC Window}
In high-speed Ethernet, the RS-FEC decoder requires a block of symbols to correct errors. For example, in 400G/800G standards, this latency can exceed hundreds of nanoseconds.
Let $T_{FEC}$ be the latency of the FEC decoder.
Let $T_{Gov}$ be the latency of the governance policy lookup.

In a traditional post-MAC model, the total latency $T_{total}$ is:
\begin{equation}
    T_{total} = T_{FEC} + T_{MAC} + T_{Gov}
\end{equation}

Our proposed **Overlap Execution** model places the governance logic in parallel:
\begin{equation}
    T_{total} = \max(T_{FEC}, T_{Gov}) + T_{MAC\_Gate}
\end{equation}
Since $T_{Gov} \approx 1.18 \mu s$ (as shown in our evaluation) and typically $T_{FEC} \gg 0$ for deep interleaving modes, the effective governance cost approaches zero marginal latency.

\subsection{PMA-PCS Anchor Point}
The governance anchor is situated immediately after the PMA sublayer. It inspects the incoming bitstream for the AGH pattern before full PCS alignment. This requires the AGH to be robustly detectable even in the presence of pre-FEC bit errors, necessitating a design that relies on statistical header recognition or specialized preamble markers.

\section{Implementation Artifacts}
To realize this PHY-level anchor, we leverage three specific implementation artifacts: the Active Governance Header (AGH), the Multi-stage Bit Vector (MsBV) lookup, and the Fisher Information Matrix (FIM) for policy stability.

\subsection{Active Governance Header (AGH)}
The AGH is a fixed-length metadata block inserted into the packet preamble or initial payload bytes. It contains a \texttt{policy\_id} that maps to a specific forwarding or dropping rule. Crucially, its fixed position and length allow for hardware parsing without complex state machines.

\subsection{MsBV: Deterministic Lookup}
To ensure $T_{Gov}$ is constant and predictable (jitter-free), we employ a Multi-stage Bit Vector (MsBV) mechanism. Unlike Ternary Content Addressable Memory (TCAM) or software hash tables which can have variable access times, MsBV uses a direct indexing approach.
The lookup complexity is $O(1)$, dependent only on the bit-width of the \texttt{policy\_id}, not the number of active policies. This guarantees that $T_{Gov}$ remains stable regardless of the control plane scale.

\subsection{Upper-Layer Policy Generation (RLCP)}
While the PHY enforcement is static and deterministic, the policies themselves are generated by an upper-layer control plane. We propose an optional Reinforcement Learning Compliance Protocol (RLCP) to dynamically generate these policies based on network conditions.
To prevent "policy churn" from destabilizing the PHY lookup tables, RLCP uses a topology-preserving loss function with a Fisher Information Matrix (FIM) constraint:
\begin{equation}
    L_{total} = L_{perf} + \gamma L_{FIM}(P_{\theta} || P_{anchor})
\end{equation}
This ensures that updates to the governance table ($\theta$) do not violate the structural integrity required for deterministic MsBV lookups.

\section{Evaluation}
We evaluate the proposed mechanism's ability to provide deterministic control within the constraints of a high-speed PHY.

\subsection{Deterministic Latency Analysis}
We subjected the MsBV interceptor to a flood of 1,000,000 packets to measure processing variance.
\begin{itemize}
    \item \textbf{Mean Latency ($T_{Gov}$)}: 1.1773 $\mu$s
    \item \textbf{Jitter}: $< 0.01 \mu$s
    \item \textbf{Throughput}: $> 800$ kpps (software emulation), scaling linearly in hardware.
\end{itemize}

As shown in Figure \ref{fig:perf}, the MsBV mechanism maintains a flat latency profile (Red Line) regardless of the number of policy rules ($10^1$ to $10^6$), contrasting sharply with the linear degradation of traditional ACLs (Gray Line). This stability is critical for ensuring $T_{Gov}$ never exceeds the $T_{FEC}$ window.

\begin{figure}[h]
    \centering
    \includegraphics[width=\columnwidth]{figures/autonomous_performance.pdf}
    \caption{Latency vs Policy Scale: MsBV+ maintains $O(1)$ lookup time, essential for overlap execution.}
    \label{fig:perf}
\end{figure}

\subsection{Convergence and Stability}
The RLCP policy generator was tested for stability. Figure \ref{fig:loss_curve} shows the loss convergence, demonstrating that the control plane can adapt to new constraints without inducing oscillations that would disrupt the PHY-level tables.

\begin{figure}[h]
    \centering
    \includegraphics[width=\columnwidth]{figures/loss_curve.pdf}
    \caption{Control Plane Stability: RLCP convergence ensures stable policy updates.}
    \label{fig:loss_curve}
\end{figure}

\section{Conclusion}
This paper presents a cross-layer control architecture for IEEE 802.3dj networks. By anchoring governance at the PHY level and employing overlap execution within the RS-FEC decode window, we eliminate the serialization penalty of traditional L3 enforcement. The MsBV mechanism provides the necessary $O(1)$ deterministic latency to make this viable. This approach paves the way for ultra-low latency, jitter-free Ethernet suitable for next-generation industrial and time-sensitive applications.

\bibliographystyle{IEEEtran}
\bibliography{refs}

\end{document}
